\documentclass[11pt]{report}
\usepackage{fullpage}
\usepackage{minted}
\usepackage{graphicx}
\usepackage{hyperref}
\usepackage{keystroke}
\usepackage{url}



\graphicspath{{./images/}{./Student Documentation/Guides/VSCode Install/images/}}

\newcommand{\WindowsLogo}{\raisebox{-0.1em}{
  \includegraphics[height=0.8em]{Windows-3-logo-simplified.png}}}
\newcommand{\WinKey}{\keystroke{\WindowsLogo}}

\newcommand*{\Cmd}{\keystroke{Cmd}}

\hypersetup{
    colorlinks=true,
    linkcolor=blue,
    filecolor=magenta,      
    urlcolor=cyan,
    pdftitle={VSCode Setup},
    pdfpagemode=FullScreen,
    }


\begin{document}
    \title{APSC 258: Setup VSCode For Developing Machine Learning Projects}
    \author{Andre Cox}


    \maketitle
    \tableofcontents

    \chapter{Introduction}
    
    \section{The Problem}


    When working with machine learning projects you will need an editor that can allow you to easily edit and run your code.
    You may choose to use Google Colab however recent versions of Colab have limited the GPU compute time and always need to be online.
    A good option for machine learning projects is VSCode. VSCode is a free open source editor that can be installed on your computer that supports almost all programming languages including Python.
    dir: \CurrentFile
    \section{Prerequisites}

    To install VSCode you will need a Windows, Mac or Linux computer. This install guide will be split into 3 parts for each operating system.


    \chapter{Installation}
    \section{Windows}

    To install VSCode on Windows follow the steps below.

    \begin{itemize}
        \item Press the keyboard shortcut \WinKey + \keystroke{R} 
        \item Type in "cmd" and press \Enter
        \item A new window should open. Type in "winget install -e --id Microsoft.VisualStudioCode" and press \Enter
        \item Once this finishes VSCode should be installed.    
    \end{itemize}

    \section{Mac}
    \begin{itemize}
        \item You can install VSCode by following the instructions on the website. Found \href{https://code.visualstudio.com/docs/setup/mac}{here}.
    \end{itemize}

    \section{Linux}
        \begin{itemize}
            \item You can install VSCode by following the instructions on the website. Found \href{https://code.visualstudio.com/docs/setup/linux}{here}.
            \item You may be able to use apt, yum or pkg to install VSCode. This depends if it's in your package repository.
        \end{itemize}

    \chapter{Testing}

    Once installed you should be able to open VSCode and start editing your code. VSCode has built in support for Jupyter Notebooks just like Google Colab.

\end{document}