\documentclass[11pt]{report}
\usepackage{fullpage}
%\usepackage{sourcesanspro, sourcecodepro}
\usepackage{minted}
\usepackage{graphicx}
\usepackage{awesomebox}
\usepackage{hyperref}
\usepackage[a4paper, total={6in, 8in}, margin=0.75in]{geometry}
\usepackage{etoolbox}
\makeatletter
\patchcmd{\chapter}{\if@openright\cleardoublepage\else\clearpage\fi}{}{}{}
\RequirePackage[T1]{fontenc}
\RequirePackage[default,light,black]{roboto}

\hypersetup{
    colorlinks=true,
    linkcolor=blue,
    citecolor=blue,
    filecolor=blue,
    urlcolor=blue,
    pdfborder={0 0 0}
}

\graphicspath{{./images/}}

\title{APSC 258: Lab 7 Manual}
\author{Andre Cox \\ Scott Halston}

\begin{document}
\maketitle
\tableofcontents

% page break
\clearpage

\chapter{Introduction}
\section{Introduction}
In our last lab, you were shown how to use droppout layers in your neural network. In combining droppout layers along with your previous knowlage of convolution layers and dense layers, hopefully you have been successfull in creating a neural network that can accuratly and quickly follow lanes.
\\ \\
In this lab, we will only be discussing the measures and parameters that will be used to grade your model. The intention for this lab is that it will give you a clear outline of what is expected from your neural network.

\chapter{Big Three}
The "Big Three" things that your models will be graded on are as follows:
    Accuracy, Speed, and Size.
Your models will be run through a script which will record your MSE values and the size of your model.

\section{Accuracy}
Accuracy will be determined through 




\end{document}
